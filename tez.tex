\documentclass[12pt]{etutez}

%% DEĞİŞKENLER
\baslik{TÜRKÇE BAŞLIK}
\title{TITLE OF THE THESIS}
\anahtarlar{anahtar kelimeler}
\keywords{keywords}
\teztipi{DOKTORA TEZİ}
\teztipikucuk{Doktora}
\thesistype{Ph.D.}              % Yuksek lisans icin M.Sc. doktora icin Ph.D.
\yazar{ÖRNEK ÖĞRENCİ}
\yazarkucuk{Örnek ÖĞRENCİ}        % Soyadi buyuk harflerle yazilmali
\bolum{BİLGİSAYAR MÜHENDİSLİĞİ}
\bolumkucuk{Bilgisayar Mühendisliği}
\dept{Computer Engineering}
\enstitu{FEN BİLİMLERİ ENSTİTÜSÜ}
\enstitukucuk{Fen Bilimleri Enstitüsü}
\institute{Institute of Natural and Applied Sciences}
\universite{TOBB EKONOMİ VE TEKNOLOJİ ÜNİVERSİTESİ}
\universitekucuk{TOBB Ekonomi ve Teknoloji Üniversitesi}
\university{TOBB University of Economics and Technology}
\danisman{Doç. Dr. xxx}
\advisor{Assoc. Prof. xxx}
\juribaskani{Prof. Dr. xxx}
\juriuyesibir{Yrd. Doç. Dr. xxx}
\juriuyesiiki{Doç. Dr. xxx}
\juriuyesiuc{Yrd. Doç. Dr. xxx}
\abdbaskani{Doç. Dr. xxx}
\enstitumuduru{Prof. Dr. xxx}
\tarih{EYLÜL 2015}
\tarihkucuk{Eylül 2015}
\submitdate{September 2015}

% sayfa numarası için
\fancyhf{}
\cfoot{\thepage}

%%%%%%%%%%%%%%%%%%%%%%%%%%%%%%%%%%%%%%%%%%%%%%%
%% BEGIN DOCUMENT
\begin{document}

\pagestyle{plain}
\pagenumbering{roman}

\titlepagePhD					% kapak sayfası
\signaturepagePhD				% imza sayfası
\tezbildirimsayfasi				% tez bildirim

\begin{ozet}
Doktora tezimin Türkçe Özeti budur.
\end{ozet}

\begin{abstract}
This is the abstract of my thesis.
\end{abstract}

\begin{tesekkur}
Bu çalışmayı tamamlamamda emeği geçen herkese teşekkür ederim...
\end{tesekkur}

\newpage
\addcontentsline{toc}{chapter}{\numberline{\contentsname}}
\tableofcontents

\newpage
\addcontentsline{toc}{chapter}{\numberline{\listfigurename}}
\listoffigures    % Tezde herhangi bir sekil yoksa silinmelidir

\newpage
\addcontentsline{toc}{chapter}{\numberline{\listtablename}}
\listoftables     % Tezde herhangi bir tablo yoksa silinmelidir

\newpage
\pagestyle{plain}
\addcontentsline{toc}{chapter}{\numberline{KISALTMALAR}}
\begin{center}
\textbf{KISALTMALAR}
\end{center}
\vspace{\satbos}

\begin{table}[h]
\begin{tabularx}{\textwidth}{@{}lcX@{}}
\textbf{Kısaltma} &  & \textbf{Açıklama}\\
EMD & : & Toprak taşıyıcı mesafesi (Earth Mover's Distance)\\
FCM & : & Bulanık öbekleme (Fuzzy c-means)\\
SIFT & : & Ölçeklemeden bağımsız özellik dönüşümü (Scale-invariant feature transform)\\
\end{tabularx}
\end{table}

\newpage
\pagestyle{plain}
\addcontentsline{toc}{chapter}{\numberline{SEMBOLLER}}
\begin{center}
\textbf{SEMBOLLER}
\end{center}
\vspace{\satbos}

\begin{table}[h]
\begin{tabularx}{\textwidth}{@{}lcX@{}}
\textbf{Sembol} &  & \textbf{Açıklama}\\
S & : & Speed\\
\end{tabularx}
\end{table}


\newpage
\pagenumbering{arabic}
%\setcounter{page}{1}
%\setcounter{section}{1}
%\setcounter{secnumdepth}{5}
%\setcounter{tocdepth}{4}

\chapter{GİRİŞ}

Bu bir atıftır \cite{de73art}

\chapter{LİTERATÜR TARAMASI}

\chapter{DENEYSEL ÇALIŞMA}

\chapter{DEĞERLENDİRME}

\chapter{SONUÇ}


\bibliographystyle{abbrv} % kaynakca.bib ile beraber kullanim icin.
\bibliography{kaynakca}   % kaynakca.bib dosyasi hazirlanirsa bu satir kullanilmali

\renewcommand{\appendixname}{}
\renewcommand{\appendixtocname}{EKLER}
\renewcommand{\appendixpagename}{EKLER}

\begin{appendices}
\appendix
\chapter{Veriler}

\chapter{Algoritma}

\end{appendices}

\newpage
\pagestyle{plain}
\addcontentsline{toc}{chapter}{\numberline{ÖZGEÇMİŞ}}
\begin{center}
\textbf{ÖZGEÇMİŞ}
\end{center}
\vspace{\satbos}

\textbf{Kişisel Bilgiler}\\
\begin{tabular}{@{}lll@{}}
Soyadı, Adı & : & ÖĞRENCİ, Örnek\\
Uyruğu & : & T.C.\\
Doğum tarihi ve yeri & : & xx.xx.xxx, Ankara\\
Medeni hali & : & xxx \\
Telefon & : & (0 xxx) xxx xx xx \\
Faks & : &\\
E-posta & : & xxx@etu.edu.tr\\
\end{tabular}
\vspace{\satbos}

\textbf{Eğitim}\\
\begin{tabular}{@{}lll@{}}
\textbf{Derece} & \textbf{Eğitim Birimi} & \textbf{Mezuniyet Tarihi}\\
Y. Lisans & XXXX Üniversitesi / Matematik & 20XX\\
Lisans & XXXX Üniversitesi / Matematik & 20XX\\
\end{tabular}
\vspace{\satbos}

\textbf{İş Deneyimi}\\
\begin{tabular}{@{}lll@{}}
\textbf{Yıl} & \textbf{Yer} & \textbf{Görev}\\
2013- & XXXXX Kurumu & Matematikçi\\
2008-2013 & TOBB Ekonomi ve Teknoloji Üniversitesi & Araştırma Görevlisi\\
\end{tabular}
\vspace{\satbos}

\textbf{Yabancı Dil}\\
\begin{tabular}{@{}l@{}}
İngilizce (Çok İyi)\\
Fransızca (İyi)\\
\end{tabular}
\vspace{\satbos}

\textbf{Yayınlar}\\
Öğrenci, Ö., Örnek makale başlığı. International Conference on XXXXXXX. 2013.\\


\end{document}
