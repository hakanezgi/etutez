%% ETÜ TEZ şablonu yüklenir
\documentclass[12pt]{etutez}

% Tez ile ilgili değişkenleri (öğrenci ad, soyad, bilim dalı, hocalar, vb.),
% özet, abstract ve teşekkür kısmını bu dosyaya giriyoruz

%% DEĞİŞKENLER
\baslik{TÜRKÇE BAŞLIK}
\title{TITLE OF THE THESIS}
\anahtarlar{anahtar kelimeler}
\keywords{keywords}
\teztipi{DOKTORA TEZİ}
\teztipikucuk{Doktora}
\thesistype{Ph.D.}              % Yüksek lisans için M.Sc. doktora için Ph.D.
\yazar{ÖRNEK ÖĞRENCİ}
\yazarkucuk{Örnek ÖĞRENCİ}      % Soyadı büyük harflerle yazılmalı
\bolum{BİLGİSAYAR MÜHENDİSLİĞİ}
\bolumkucuk{Bilgisayar Mühendisliği}
\dept{Computer Engineering}
\enstitu{FEN BİLİMLERİ ENSTİTÜSÜ}
\enstitukucuk{Fen Bilimleri Enstitüsü}
\enstitukisa{Fen Bilimleri}
\institute{Institute of Natural and Applied Sciences}
\universite{TOBB EKONOMİ VE TEKNOLOJİ ÜNİVERSİTESİ}
\universitekucuk{TOBB Ekonomi ve Teknoloji Üniversitesi}
\university{TOBB University of Economics and Technology}
\danisman{Doç. Dr. xxx}
\advisor{Assoc. Prof. xxx}
\juribaskani{Prof. Dr. xxx}
\juriuyesibir{Yrd. Doç. Dr. xxx}
\juriuyesiiki{Doç. Dr. xxx}
\juriuyesiuc{Yrd. Doç. Dr. xxx}
\abdbaskani{Doç. Dr. xxx}
\enstitumuduru{Prof. Dr. xxx}
\tarih{EYLÜL 2015}
\tarihkucuk{Eylül 2015}
\submitdate{September 2015}

%% Sayfa numarası için (sayfa altına, ortalanmış)
\fancyhf{}
\cfoot{\thepage}

%% TEZ BAŞLIYORRRRR
\begin{document}

\pagestyle{plain}
\pagenumbering{roman}

%% KAPAK, İMZA, TEZ BİLDİRİM
\kapaksayfasi
\imzasayfasi
\tezbildirimsayfasi

%% ÖZET SAYFASI
\begin{ozet}
Doktora tezimin Türkçe Özeti budur.
\end{ozet}

%% ABSTRACT SAYFASI
\begin{abstract}
This is the abstract of my thesis.
\end{abstract}

%% TEŞEKKÜR SAYFASI
\begin{tesekkur}
Bu çalışmayı tamamlamamda emeği geçen herkese teşekkür ederim...
\end{tesekkur}

%% İÇİNDEKİLER
\newpage
\addcontentsline{toc}{chapter}{\numberline{\contentsname}}
\tableofcontents

%% ŞEKİLLER LİSTESİ
% Tezde herhangi bir şekil yoksa çıkartılmalıdır
\newpage
\addcontentsline{toc}{chapter}{\numberline{\listfigurename}}
\listoffigures

%% TABLOLAR LİSTESİ
% Tezde herhangi bir tablo yoksa çıkartılmalıdır
\newpage
\addcontentsline{toc}{chapter}{\numberline{\listtablename}}
\listoftables

%% KISALTMALAR ve SEMBOLLER
\newpage
\pagestyle{plain}
\addcontentsline{toc}{chapter}{\numberline{KISALTMALAR}}
\begin{center}
\textbf{KISALTMALAR}
\end{center}
\vspace{\satbos}

\begin{tabular}{@{}lll@{}}
\textbf{Kısaltma} &  & \textbf{Açıklama}\\
EMD & : & Earth Mover's Mesafesi\\
FCM & : & Bulanık öbekleme (Fuzzy c-means)\\
\end{tabular}

\newpage
\pagestyle{plain}
\addcontentsline{toc}{chapter}{\numberline{SEMBOLLER}}
\begin{center}
\textbf{SEMBOLLER}
\end{center}
\vspace{\satbos}

\begin{table}[h]
\begin{tabularx}{\textwidth}{@{}lcX@{}}
\textbf{Sembol} &  & \textbf{Açıklama}\\
S & : & Speed\\
\end{tabularx}
\end{table}


%% TEZ BÖLÜMLERİ
\newpage
\pagenumbering{arabic}

\chapter{GİRİŞ}

Bu bir atıftır \cite{de73art}

\input{bolum2}
\input{bolum3}
\input{bolum4}
\input{bolum5}

%% KAYNAKLAR
\bibliographystyle{abbrv}  % .bib dosyası ile beraber kullanım için
\bibliography{kaynaklar}   % .bib dosyası hazırlanırsa bu satır kullanılmalı

%% EKLER
\begin{appendices}
\appendix
\input{ek1}
\input{ek2}
\end{appendices}

%% ÖZGEÇMİŞ
\newpage
\pagestyle{plain}
\addcontentsline{toc}{chapter}{\numberline{ÖZGEÇMİŞ}}
\begin{center}
\textbf{ÖZGEÇMİŞ}
\end{center}
\vspace{\satbos}

\textbf{Kişisel Bilgiler}\\
\begin{tabular}{@{}lll@{}}
Soyadı, Adı & : & ÖĞRENCİ, Örnek\\
Uyruğu & : & T.C.\\
Doğum tarihi ve yeri & : & xx.xx.xxx, Ankara\\
Medeni hali & : & ______ \\
Telefon & : & (0 xxx) xxx xx xx \\
Faks & : &\\
E-posta & : & _______@etu.edu.tr\\
\end{tabular}
\vspace{\satbos}

\textbf{Eğitim}\\
\begin{tabular}{@{}lll@{}}
\textbf{Derece} & \textbf{Eğitim Birimi} & \textbf{Mezuniyet Tarihi}\\
Y. Lisans & _______ Üniversitesi / Matematik & 20__\\
Lisans & _______ Üniversitesi / Matematik & 20__\\
\end{tabular}
\vspace{\satbos}

\textbf{İş Deneyimi}\\
\begin{tabular}{@{}lll@{}}
\textbf{Yıl} & \textbf{Yer} & \textbf{Görev}\\
2013- & _______ Kurumu & Yazılımcı\\
2008-2013 & TOBB Ekonomi ve Teknoloji Üniversitesi & Araştırma Görevlisi\\
\end{tabular}
\vspace{\satbos}

\textbf{Yabancı Dil}\\
\begin{tabular}{@{}l@{}}
İngilizce (Çok İyi)\\
Fransızca (İyi)\\
\end{tabular}
\vspace{\satbos}

\textbf{Yayınlar}\\
Öğrenci, Ö., Örnek makale başlığı. International Conference on _______. 2013.\\


\end{document}
%% TEZ BİTTİ :)
